\subsection{The parabola is continuous}

Sei $f(x) = x^2$. Wir beweisen, dass $f$ an jeder Stelle $x \in \mathbb{R}$ stetig ist. 
\textbf{Vorbereitende Schritte:}

1. \textbf{Definition von $\delta$:}
   \[
   \delta = \min\left(\frac{\epsilon}{2|x| + 1}, 1\right)
   \]
   Diese Wahl von $\delta$ stellt sicher, dass $\delta > 0$ und $\delta \leq 1$.

2. \textbf{Positivität von $\delta$:}
   \[
   0 < \delta
   \]
   Da $\epsilon > 0$ und $2|x| + 1 > 0$, folgt $0 < \frac{\epsilon}{2|x| + 1}$. Somit ist $\delta > 0$.

3. \textbf{Obere Schranke von $\delta$:}
   \[
   \delta \leq 1
   \]
   Dies folgt direkt aus der Definition von $\delta$.

4. \textbf{Weitere obere Schranke von $\delta$:}
   \[
   \delta \leq \frac{\epsilon}{2|x| + 1}
   \]
   Auch dies folgt direkt aus der Definition von $\delta$.

5. \textbf{Beziehung zwischen $|y|$ und $|x|$:}
   \[
   |y| < |x| + \delta
   \]
   Da $|y| = |x + (y - x)| \leq |x| + |y - x|$ und $|y - x| < \delta$, folgt $|y| < |x| + \delta$.

6. \textbf{Beziehung zwischen $|x + y|$ und $|x| + |y|$:}
   \[
   |x + y| \leq |x| + |y|
   \]
   Dies ist eine Anwendung der Dreiecksungleichung.

7. \textbf{Nichtnegativität von $|x - y|$:}
   \[
   0 \leq |x - y|
   \]
   Da $|x - y|$ der Betrag einer reellen Zahl ist, ist er nicht negativ.

8. \textbf{Obere Schranke von $|x - y|$:}
   \[
   |x - y| \leq \delta
   \]
   Da $|y - x| < \delta$, folgt $|x - y| = |y - x| < \delta$.

9. \textbf{Beziehung zwischen $|x| + |y|$ und $|x| + (|x| + \delta)$:}
   \[
   |x| + |y| < |x| + (|x| + \delta)
   \]
   Da $|y| < |x| + \delta$, folgt $|x| + |y| < |x| + (|x| + \delta)$.

10. \textbf{Beziehung zwischen $2|x| + \delta$ und $2|x| + 1$:}
    \[
    2|x| + \delta \leq 2|x| + 1
    \]
    Da $\delta \leq 1$, folgt $2|x| + \delta \leq 2|x| + 1$.

\textbf{Beweis:}

Sei $\epsilon > 0$ gegeben. Wir wählen $\delta$ als $\delta = \min\left(\frac{\epsilon}{2|x| + 1}, 1\right)$. Nach den obigen vorbereitenden Schritten wissen wir, dass $0 < \delta$.

Sei $|y - x| < \delta$. Wir müssen zeigen, dass $|y^2 - x^2| < \epsilon$.

\begin{align*}
|y^2 - x^2| &= |(y + x)(y - x)| & \text{(Ringregel)} \\
            &= |y + x| \cdot |y - x| & \text{(Absorptionsregel)} \\
            &\leq (|x| + |y|) \cdot |y - x| & \text{(Anwendung von Schritt 6)} \\
            &\leq (|x| + (|x| + \delta)) \cdot \delta & \text{(Anwendung von Schritt 5 und 8)} \\
            &= (2|x| + \delta) \cdot \delta \\
            &\leq (2|x| + 1) \cdot \delta & \text{(Anwendung von Schritt 10)} \\
            &\leq (2|x| + 1) \cdot \frac{\epsilon}{2|x| + 1} & \text{(Anwendung von Schritt 4)} \\
            &= \epsilon & \text{(Feldregel)}
\end{align*}

Daraus folgt, dass $|y^2 - x^2| < \epsilon$.

\hfill $\blacksquare$
