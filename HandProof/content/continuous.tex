\section{Continuous Functions}

\textit{Text excerpts remixed from Vladimir A. Zorich - Mathematical Analysis I as well as Stephen Abbott - Understanding Analysis.}

Let $f$ be a real-valued function defined in a neighborhood of a point $a \in \R$. In intuitive terms, the function $f$ is continuous at $a$ if its value $f(x)$ approaches the value $f(a)$ that it assumes at the point $a$ itself as $x$ gets nearer to $a$.

\begin{definition}[Continuous at a point]
    A function $f: D\subseteq \R \to \R$ is \textit{continuous at the point} $a \in D$ if
    \begin{align}
        \forall \epsilon > 0 \quad \exists \delta > 0
        \quad \forall x \in D: \quad
        \Bigl(|x - a| < \delta \: \implies \: |f(x) - f(a)| < \epsilon \Bigr)
    \end{align}

    If $f$ is continuous at every point in the domain $D$, then we say that $f$ is \textit{continuous on} $D$.
\end{definition}

\begin{definition}[Fibration]
    test
\end{definition}