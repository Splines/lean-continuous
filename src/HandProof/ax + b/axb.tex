\documentclass{article}
\usepackage{amsmath}
\usepackage{amssymb}

\begin{document}

\section*{Lean Code Documentation}

Gegeben sei die Funktion $f(y) = a \cdot y + b$, wobei $a \neq 0$. Wir wollen zeigen, dass $f$ stetig an jeder Stelle $x \in \mathbb{R}$ ist.
\textbf{Schritte im Detail:}

1. \textbf{Einführung:}
   Wir beginnen mit der Definition der Funktion $f(y) = a \cdot y + b$ und der Annahme $a \neq 0$. Ziel ist es zu zeigen, dass $f$ an der Stelle $x$ stetig ist. 

2. \textbf{Einführung der $\varepsilon$-Umgebung:}
   Sei $\varepsilon > 0$. Wir wählen $\delta = \frac{\varepsilon}{|a|}$.
   \[
   \text{Sei } \delta := \frac{\varepsilon}{|a|}.
   \]

3. \textbf{Existenz von $\delta$:}
   Da $|a| > 0$, folgt $\delta > 0$.
   \[
   0 < \delta := \frac{\varepsilon}{|a|}.
   \]

4. \textbf{Definition der $\delta$-Umgebung:}
   Wir zeigen nun, dass für alle $y$ mit $|y - x| < \delta$ die Bedingung $|f(x) - f(y)| < \varepsilon$ erfüllt ist. 

5. \textbf{Berechnung der Differenz:}
   \begin{align*}
   |f(x) - f(y)| &= |(a \cdot x + b) - (a \cdot y + b)| \\
   &= |a \cdot x + b - a \cdot y - b| \\
   &= |a \cdot x - a \cdot y| \\
   &= |a \cdot (x - y)| \\
   &= |a| \cdot |x - y|.
   \end{align*}

6. \textbf{Schätzung der Differenz:}
   Da $|x - y| < \delta$ und $\delta = \frac{\varepsilon}{|a|}$, erhalten wir:
   \[
   |a| \cdot |x - y| < |a| \cdot \delta = |a| \cdot \frac{\varepsilon}{|a|} = \varepsilon.
   \]

7. \textbf{Abschluss:}
   Somit haben wir gezeigt, dass $|f(x) - f(y)| < \varepsilon$ für $|x - y| < \delta$, was die Stetigkeit von $f$ an $x$ beweist.

\end{document}
