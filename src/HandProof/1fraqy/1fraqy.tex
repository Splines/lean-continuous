\documentclass{article}
\usepackage{amsmath, amssymb}
\usepackage{amsthm}

\begin{document}
\begin{proof}
Sei \( x \in \mathbb{R} \) mit \( x \neq 0 \) und \( \epsilon > 0 \) gegeben. Wir müssen ein \( \delta > 0 \) finden, so dass für alle \( y \) mit \( 0 < |y - x| < \delta \) gilt, dass \( \left| \frac{1}{y} - \frac{1}{x} \right| < \epsilon \).

Setze \( \delta = \min \left( \frac{\epsilon |x|^2}{2}, \frac{|x|}{2} \right) \).

\begin{itemize}
    \item Da \( \epsilon > 0 \) und \( |x| > 0 \), ist \( \delta > 0 \).
\end{itemize}

Sei nun \( y \) mit \( y \neq 0 \) und \( |y - x| < \delta \) gegeben.

\begin{itemize}
    \item Zuerst zeigen wir, dass \( \left| \frac{1}{x} - \frac{1}{y} \right| = \left| \frac{y - x}{xy} \right| \):
    \begin{align*}
    \left| \frac{1}{x} - \frac{1}{y} \right| &= \left| \frac{y - x}{xy} \right| \\
    &= \frac{|y - x|}{|x||y|}
    \end{align*}

    \item Da \( |y - x| < \delta \leq \frac{|x|}{2} \), folgt \( |y| > \frac{|x|}{2} \):
    \begin{align*}
    |y| &= |x + (y - x)| \\
    &\geq |x| - |y - x| \\
    &> |x| - \frac{|x|}{2} \\
    &= \frac{|x|}{2}
    \end{align*}

    \item Da \( \delta \leq \frac{\epsilon |x|^2}{2} \), folgt:
    \begin{align*}
    \frac{|x - y|}{|x||y|} &< \frac{\delta}{|x| \cdot \frac{|x|}{2}} \\
    &= \frac{\delta}{\frac{|x|^2}{2}} \\
    &\leq \frac{\frac{\epsilon |x|^2}{2}}{\frac{|x|^2}{2}} \\
    &= \epsilon
    \end{align*}
\end{itemize}

Somit haben wir gezeigt, dass für alle \( y \) mit \( y \neq 0 \) und \( |y - x| < \delta \) gilt, dass \( \left| \frac{1}{y} - \frac{1}{x} \right| < \epsilon \). Daher ist \( f \) stetig an \( x \).
\end{proof}

\end{document}
